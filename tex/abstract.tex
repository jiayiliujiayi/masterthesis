%# -*- coding: utf-8-unix -*-
% !TEX program = xelatex
% !TEX root = ../thesis.tex
% !TEX encoding = UTF-8 Unicode
%%==================================================
%% abstract.tex for SJTU Master Thesis
%%==================================================

\begin{abstract}
  \textbf{肠道菌群纵向模式与早产儿坏死性小肠结肠炎研究}
    \begin{description}
      \item[目的] 探究早产儿出生后肠道菌群定植模式与坏死性小肠结肠炎(NEC)和迟发型败血症(LOS)的相关性。
      \item[方法] 入组24名早产新生儿,其中4名在住院后发展为NEC,3名为LOS,其余17名为对照组。从出生后的第一天开始收集粪便标本,纵向收集直至患儿出院,共收集192个粪便样本。扩增每个粪便样品的16s rRNA基因的细菌V3~V4区域并测序;后续分析包括:OTU聚类,各分类水平注释,多样性分析和多级物种判别分析。
      \item[结果] 从出生后第14天开始,肠道微生物群定植模式开始在NEC,LOS及其匹配的对照组之间出现差异。迟发性脓毒症婴儿的肠道微生物群最少(Shannon指数= 1.66),对照组保持最多样化(Shannon指数= 0.88,p = 0.01)。潜在致病菌属肠球菌(20.86%)和葡萄球菌(8.67%)在NEC患者中显着,而在LOS组中克雷伯氏菌(42.15%)。两组均较对照组(3.66%)含有较少的乳球菌(NEC和LOS组分别为7.98%和13.76%)。
      \item[结论] 肠道菌群定植失调与早产儿NEC和LOS发生和发展相关,尤其是潜在致病菌属,包括链球菌、葡萄球菌和克雷伯菌的相对丰度增加与上述疾病发展相关。
    \end{description}

  \textbf{儿科炎症性疾病与肠道菌群紊乱}
    \begin{description}
      \item[目的] 详细内容
      \item[方法] 详细内容
      \item[结果] 详细内容
      \item[结论] 详细内容 \ldots
    \end{description}
\end{abstract}

\begin{englishabstract}
  \textbf{Patterened Intestinal Microbiota in Necrotizing Enterocolitis}
    \begin{description}
      \item[AIM] 详细内容
      \item[METHODS] 详细内容
      \item[RESULTS] 详细内容
      \item[CONCLUSIONS] 详细内容 \ldots
    \end{description}

  \textbf{Pediatric Enterocolitis in Association with Intestinal Microbiota}
    \begin{description}
      \item[AIM] 详细内容
      \item[METHODS] 详细内容
      \item[RESULTS] 详细内容
      \item[CONCLUSIONS] 详细内容 \ldots
    \end{description}
\end{englishabstract}
