%# -*- coding: utf-8-unix -*-
% !TEX program = xelatex
% !TEX root = ../thesis.tex
% !TEX encoding = UTF-8 Unicode
\chapter{综述\quad 肠道菌群研究中粪便样品取样与保存方法}
\label{app:review}

\begin{center} \textbf{Sampling and Storage Methods of Fecal samples \\ in human intestinal microbiome study}
\end{center}

\begin{description}
    \item[摘要] 随着高通量测序技术(next generation sequencing, NGS )的飞速发展,肠道菌群得以被更深入的研究。肠道内容物和粪便样品有着取样便捷、代表性强等特点,因此常被作为肠道菌群研究中的主要研究对象;而样品的收集和保存方法很大程度上影响其内部菌群结构和多样性,从而决定了后续测序分析的准确性。该文将就现今常用的肠道菌群研究中粪便样品的取样和保存方法进行综述。
    \item[关键词] 肠道菌群;高通量测序;肠道内容物;粪便;取样
    \item[基金项目] 国家自然科学基金(项目编号81771630)
    \item[Abstract]With the rapid development of high-throughput sequencing (NGS) technique, intestinal microbiome could be studied more deeply. Intestinal contents and fecal samples have the characteristics of convenient sampling and strong representation, so they are often used as the main research objects in the study of intestinal microbiota. The method of collecting and storage of samples are very important to affect the internal flora structure and diversity, which determines the accuracy of subsequent sequencing analysis. This review summarizes the sampling and storage methods of fecal samples in the study of intestinal microbiome.
    \item[Key Words]Intestinal Microbiome;High Throughput DNA Sequencing;Intestinal Contents;Feces;Specimen Collection
    \item[Fundings]National Natural Science Foundation of China (Project No. 81771630)
\end{description}


肠道菌群是广泛存在于人类和其他动物(包括昆虫)的肠道中的微生物所组成的复杂群落;在正常人体肠道中,存在至少1000种菌种。肠道菌群宏基因组是肠道菌群的所有基因组的总和,它包含了超过三百万个基因\cite{Saxena2016}。肠道与人体内其他部位相比,细菌数量最多、物种丰度最高\cite{Willey2013}。

越来越多的研究表明,人体肠道菌群的早期定植、构成、转换模式、代谢特点和免疫应答模式与肥胖、糖尿病、炎症性肠病、自闭症等多种疾病的发生与发展相关\cite{Li2016c,lozupone2012diversity}。因此,肠道菌群研究对于疾病发病机制及临床防治的重要性不言而喻。粪便和肠内容物有采样便捷、样品易获取和代表性强等优势,因此常被选择作为肠道菌群取样的来源\cite{Saxena2016,Raoult2014,Gill2006}。随着高通量测序技术(Next Generation Sequencing, NGS)的推广和普及,相比以往基于细菌培养的分析手段,肠道菌群的组成得以更准确、更全面、更高效地分析\cite{Goodwin2016,Fraher2012a}。然而,测序方法有着一定的敏感性,而不适当的样品采集或储存易导致实验结果的偏差,因此分析结果的准确性很大程度上取决于样品在DNA提取之前的完整性和稳定性。例如,反复冻融过程中形成的冰晶体会促使细胞破裂,导致DNA损伤和细胞凋亡,影响后续测序分析结果\cite{Fraser2011}。为保护粪便样品DNA和RNA所采取的防护措施,应以维持细胞膜稳定、保持基因活性和微生物多样性稳定为前提。近年来,一些研究比较了测序前的样品收集和制备方法对于测序和分析结果的影响。本文就现今常用的肠道菌群研究中粪便样品的取样和保存方法进行综述,以期为肠道菌群研究提供方法学参考。

\section{粪便样品取样与保存常用方法}

目前,粪便样品的统一取样与保存方法“标准操作规则”尚未统一;人类微生物组计划(Human Microbiome Project, HMP)提供的标准流程和国际人类微生物组标准(International Human Microbiome Standards, IHMS)项目提供的取样标准操作规程被作为现今肠道菌群研究的常用方法\cite{li2016characterization,Mai2011a,Chu2017,Schmidt2017}。

\subsection{HMP标准流程(HMP Core Microbiome Sampling, Protocol A)}
\subsubsection{样品采集方法}
获得患者(或监护人)的理解与许可并签署知情同意书。准备干冰盒、无菌手套、无菌塑料铲、无菌采样管。带无菌手套,使用无菌塑料勺,收集粪便或肠内容物样品(约1g)于无菌采样管内,样本进行编号后立刻放入干冰盒中\cite{wu2018optimization}。
\subsubsection{样品保存条件}
反复冻融后冰晶会微生物细胞破裂、DNA裂解,故应尽量立即提取样品DNA。若无立即提取DNA的条件,则应在样品置于干冰盒后30 min内,将其转移至冻存管内,并于-80℃条件下冻存。若样品后续需提取RNA, 则应在每500μl样品中加入1ml RNAlater\textsuperscript{\textregistered} (Ambion, Inc., AM7020)稳定液,转移至冻存管内,于-80℃条件下冻存\cite{wu2018optimization}。

\subsection{IHMS标准操作规程(IHMS Sample Collection and Handling SOPs)}
\subsubsection{样品采集方法}
获得患者(或监护人)的理解与许可并签署知情同意书。准备干冰盒、无菌手套、无菌塑料铲、无菌采样管。带无菌手套,使用无菌塑料勺,收集粪便或肠内容物样品(约1g)于无菌采样管内,样本进行编号后立刻放入干冰盒中\cite{santiago2014processing}。
\subsubsection{样品保存条件}
\begin{enumerate}
  \item 若样品能够在采集后4小时内送至实验室进行DNA或RNA提取,则可将其室温储存并立即运输。若样品能够在采集后4~24小时内送至实验室进行DNA或RNA提取,则需在室温条件下使用Anaerocult\textsuperscript{\textregistered} 厌氧培养袋储存样品并尽快送至实验室。
  \item 若样品能够在采集后1~7天内送至实验室进行DNA或RNA提取,则需将其置于干冰盒内,或加入相应稳定液,并使用Anaerocult\textsuperscript{\textregistered}(Merck Millopore, Germany)厌氧培养袋储存,7天内转移至-80℃条件下冻存。
\end{enumerate}

\section{粪便样品取样方法对实验结果的影响}
由于环境和(或)实验条件的限制,肠道菌群研究中的取样方法可能并非完全遵守上述取样标准;收集替代样品、取样温度的差异和使用试剂的不同等多种因素均可能对研究结果产生影响。以下总结不同粪便取样方法对于后续测序分析结果的影响。
\subsection{取样时间}
应在粪便排出后2 h内尽快取样,防止需氧菌和兼性厌氧菌过度生长,导致菌群丰度产生偏移\cite{santiago2014processing}。对于使用尿布的小儿患者,应每隔0.5~1 h检测是否排便以便尽快取样\cite{shaw2016latitude},防止因局部温度过高导致的菌群生长抑制或过度生长导致后续分析结果偏倚\cite{guo2016effect}。
\subsection{选择替代样品}
当粪便或肠内容物样品难以获得时,可以选择收集结肠灌洗液作为替代样品。一项纳入了23例成年人粪便标本的研究显示,结肠灌洗液菌群的$\alpha$多样性(包括OTU多样性和均匀性)和菌门相对丰度均与肠道黏膜样品菌群类似\cite{Watt2016}。
\subsection{取样人员}
当研究需纳入多地区、大样本或按时间顺序纵向进行取样时,因人力有限,有时要求研究对象代替研究人员自行取样,即研究对象自行在病房或家中收集粪便样品,再将其送至或邮寄至研究机构\cite{kumar2014getting}。研究对象自行取样时,往往缺乏存放样品所使用的干冰盒等条件;Nechvatal等\cite{nechvatal2008fecal}研究显示,自行取样和邮寄时温度等因素会影响样品后续的分析结果。为避免分析结果偏移,应尽量选择使用方法简便的试剂盒进行取样,该研究推荐自行取样后立即置入RNAlater™ (Ambion, Austin, TX)稳定剂(每0.2 g粪便样品使用1mL RNAlater™试剂),再进行样品运输,其测序分析结果$\alpha$多样性、$\beta$多样性和OTU数目等未产生较大偏移,变异性较小。因此,这种方法较适用于基于大样本量的肠道菌群流行病学研究。
\subsection{取样次数}
通过多次取样,可以降低由方法学带来的相对误差。Gorzelak等\cite{gorzelak2015methods}研究表明,对于同一研究对象的同一样品在相同取样条件下进行多次取样,再将各样品通过液氮冷冻形成细微粉末进行均质化后,测序分析得到的细菌类群丰度在各样品间的差异显著降低。部分原因是由于液氮速冻能够直接形成玻璃态,避免冰晶形成对微生物细胞内DNA造成的机械性损伤\cite{iancu2006comparison}。
\subsection{采样后、保存前所使用的试剂}
由于取样条件的限制,样品取得后可能无法立即置于干冰盒内,或在干冰盒内放置时间超过30 min,而样品所处的温度条件、氧气含量和稳定剂的使用皆可能对研究结果产生不同的影响\cite{nechvatal2008fecal,iancu2006comparison}。
Wu等\cite{wu2010sampling}研究表明,若取样后不加入稳定剂,放置于-4℃环境中48 h,后置于-80℃条件下冻存,其菌群结构和多样性的变异与取样后立即于-80℃条件下冻存样品无显著差异;若取样后加入PSP\textsuperscript{\textregistered}(Invitek GmbH, Germany)缓冲液并置于室温下48h,再冻存于-80℃条件下,则厚壁菌门的少部分菌属丰度有所增加。同样,若样品在-80℃条件下冻存前,于室温下放置24 h或更短时间,其后续测序结果相比取样后立即于-80℃冻存的样品仍未产生较大变异\cite{tedjo2015effect}。然而,也有研究显示,取样后置于室温下15~30 min后,样品内厚壁菌门丰度增加、拟杆菌门丰度降低;而取样后若样品置于家用无霜冰箱(温度范围-20~-2℃)超过3 d,其拟杆菌门丰度开始显著降低,双歧杆菌属、乳杆菌属和肠杆菌属丰度均显著降低\cite{gorzelak2015methods}。
若取样后的低温放置条件尚难达到,可以使用RNAlater\textsuperscript{TM}(Ambion, Austin, TX)等稳定剂于室温下保存。关于各种稳定剂对于菌群样品的维持作用,一项纳入了52例成年人的研究报道,使用95\%酒精作为稳定剂的样品,其$\alpha$多样性与立即冻存样品相比稍有降低,而使用粪便免疫化学试剂管收取或加入RNAlater™稳定剂等样品,其内部$\alpha$多样性值和各菌群丰度(变形菌门和拟杆菌门为主)并未发现显著改变\cite{vogtmann2017comparison}。另外,此研究与另一研究均提示,使用粪便隐血试卡FOBT Hemoccult Sensa\textsuperscript{\textregistered} (Beckman Coulter Brea, CA)取样并在常温下放置于Ep管内超过3天的样品相比立即于-80℃冻存的样品,其菌群多样性及和各菌门的丰度均无明显差异,因此FOBT收集样品适合用于需要进行远距离邮寄的样品研究\cite{vogtmann2017comparison,dominianni2014comparison}。
\section{粪便样品不同保存方法对研究结果的影响}
以往多数研究表明,储存粪便样品的条件只会轻度影响其微生物群落的结构。Shaw等\cite{shaw2016latitude}研究表明,若样品收集后存放于-80℃条件保存长达2年,其菌群分布只产生了微小变化——乳杆菌和芽孢杆菌丰度略有降低,而$\alpha$多样性和总OTU计数略减少,但无统计学意义。另外,Choo等\cite{choo2015sample}比较了分别于室温、4℃条件和-80℃条件保存的样品,以-80℃条件为标准,后两者样品$\alpha$多样性值和各菌群丰度等分析未产生显著性差异,而室温下保存的样品的放线菌门双歧杆菌属的丰度显著降低。于-20℃条件下冻存样品相比取样后立即提取DNA并扩增的样品,其厚壁菌门:拟杆菌门比值升高\cite{bahl2012freezing}。样品在室温环境下保存时间的长短对于样品微生物组成以及菌群多样性仍然存在争议\cite{guo2016effect,nishimoto2016high,carroll2012characterization}。
此外,现有的一些菌群样品保存的试剂具有较好的技术可重复性,也能够较好的在室温条件下维持菌群稳定性,其中包括OMNIgene∙GUT (DNA Genotek, Inc. Ottawa, CAN),和上文提到的RNAlater\textsuperscript{TM}(Ambion, Austin, TX)以及硫氰酸胍溶液\cite{nishimoto2016high}等。
\section{总结}
肠道菌群的研究很大程度上依赖粪便和肠内容物样品的采集。随着对肠道菌群研究的发展,亟须能够直接比较不同数据集之间结果的标准操作流程。然而,目前对于人体肠道菌群研究中的粪便样品的采集与保存方法尚未有统一的“金标准”。
因此,亟须优化并标准化收集程序和储存条件,从而减缓样品中的DNA降解、减少微生物分析中的变异,便于比较不同研究结果。在未来的研究中还应纳入更多实验样品,将人口学特征和疾病状态纳入实验设计,确定最优化的操作规则,为后续的测序及分析研究提供便利。
