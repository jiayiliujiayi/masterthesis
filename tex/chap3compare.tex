%# -*- coding: utf-8-unix -*-
% !TEX program = xelatex
% !TEX root = ../thesis.tex
% !TEX encoding = UTF-8 Unicode
%%==================================================
%% chap3compare.tex for SJTU Master Thesis
%%==================================================

%\bibliographystyle{sjtu2}%[此处用于每章都生产参考文献]
\chapter{儿科肠道炎症性疾病与肠道菌群研究}
\label{chap:compare}

\section{引言}
\section{材料与方法}
  \subsection{伦理}
  该研究经上海市儿童医学中心伦理联合委员会批准,上海交通大学医学院(SCMCIRB-K2013022)。 在采集每名患儿的粪便样本之前,取得其监护人理解、同意并于知情同意书上签字。
  \subsection{研究对象}
  于2017年3月至2018年6月间,上海儿童医学中心新生儿重症监护病房(NICU)的NEC患儿、外科病房的HD、HAEC患儿和上海市儿童医院消化科病房的IBD患儿,
    \subsubsection{NEC入选及排除标准}
      \paragraph{入组标准}
      胎龄小于34周,出生体重不低于950g的住院患儿。
      \paragraph{排除标准}
      1)新生儿早发型败血症,2)肝脏疾病,3)肾功能损害(Cr>88μM),4)存在先天肠道发育异常,5)需要进行大型胸部或腹部手术(男性包皮环切术或动脉导管结扎除外),6)预计使用肠外营养(Parental Nutrition, PN)支持供应超过50%的每日卡路里摄入量,且时间超过4天,7)静脉注射抗生素(除头孢噻肟、哌拉西他唑和甲硝唑),8)有口服抗生素史,9)有血便史,10)日龄大于五天者
    \subsection{HD和HAEC入选及排除标准}
      \paragraph{入选标准} 1)病理确诊HD对患儿,确诊HAEC的患儿;2)需要接受手术治疗HD的患儿;3)年龄0~4岁。
      \paragraph{排除标准} 1)合并其它疾病或畸形(如先天性心脏病、唐氏综合症等),2)入组前一周内接受过抗生素,3)入组前三个月内接受过益生菌制剂。
    \subsection{IBD入选及排除标准}
      \paragraph{入选标准} 根据儿童和青少年炎症性肠病的诊断标准指南\cite{levine2014espghan},初次诊断儿童克罗恩病和溃疡性结肠炎的患儿。
      \paragraph{排除标准} 曾经接受过肠切除手术的患儿。
  \subsection{诊断标准}
    坏死性小肠结肠炎(NEC)诊断和分级:关注每名患儿入院后的健康情况,评估并记录一般情况全身,腹部平片报告结果等;根据“改良Bell分级标准”\cite{bell1978neonatal}:II期,伴有放射性肠扩张,肠梗阻,肠道积气和/或腹部压痛,和/或轻度代谢性酸中毒,肠鸣音减弱,血小板减少症。

    先天性巨结肠(HD)诊断标准为Elhalaby临床诊断标准\cite{elhalaby1995enterocolitis}和Teitelbaum病理诊断标准。

    炎症性肠病(IBD)的准确诊断:基于病史,体格检查和实验室检查,食管胃十二指肠镜检查(EGD)和组织学的回肠结肠镜检查以及小肠镜的组合。



  \subsection{主要实验室试剂及仪器}
  参照\ref{主要实验室试剂及仪器}
  \subsection{粪便标本采集方法}
  参照\ref{粪便标本采集方法}
  \subsection{标本总DNA提取}
  参照\ref{标本总DNA提取}
  \subsection{总DNA 16s rDNA V3-V4可变区片段的扩增}
  参照\ref{总DNA 16s rDNA V3-V4可变区片段的扩增}
  \subsection{荧光定量}
  参照\ref{荧光定量}
  \subsection{Illumina Miseq下一代高通量测序}
  参照\ref{Illumina Miseq下一代高通量测序}
  \subsection{原始数据处理}
  参照\ref{原始数据处理}
  \subsection{统计学方法}
  参照\ref{统计学方法}
  \begin{enumerate}
    \item 使用anova和卡方检验比较患者临床基本信息。
    \item 使用Kruskal-Wallis H test检验各分类水平下各物种在各组间的分布(相对丰度)是否存在显著性差异;使用Benjaminia and Hochberg法对\textit{p}值进行多重检验矫正。使用Student’s \textit{t}检验检验各分类水平下各物种在两个组之间的分布(相对丰度)是否存在显著性差异;使用Benjaminia and Hochberg法对\textit{p}值进行多重检验矫正。
    \item 使用Student’s t检验评估两组间多样性指数的差异。若p < 0.05 则表示差异有统计学意义。过$\alpha$多样性分析可以得到群落(分组或者单个样本)中物种的丰度、覆盖度和多样性等信息。
    \item 利用Wilcoxon秩和检验,对每一组中的亚组进行两两检验,在具有显著差异物种类中的所有亚种比较是否都趋同于同一分类级别。
  \end{enumerate}

\section{结果}
  \subsection{患者基本情况}

  \subsection{样本及测序信息}
  入选HAEC患儿4例,HD患儿7例,NEC患儿6例,IBD患儿7例,共24例。采集其横向粪便标本,共192个,其中HAEC标本14个,HD标本22个,NEC标本15个,IBD标本7个。其他临床资料见表\ref{tab:comparedemographic}。

  \begin{table}[!hpb]
    \centering
    \bicaption[早产儿患者临床情况]
      {早产儿患者临床情况\footnotemark[1]}
      {Demographic characteristics of Preterm NEC, LOS and control groups.}
    \label{tab:comparedemographic}
    \begin{tabular}{lp{1.8cm}p{1.8cm}p{1.8cm}p{1.8cm}p{2cm}c}
      \toprule
         & \textbf{HAEC(n=4)} & \textbf{HD(N=7)} & \textbf{NEC(N=6)} & \textbf{IBD(N=7)} & \textbf{Statistical Test} & \textit{p value} \\ \midrule
        \textbf{年龄(天))} & 29(29-30) & 30(29-31) & 31(28-33) & Kruskal-Wallis test & 0.074 \\
        \textbf{出生体重(克)} & 1416.3 (773.4-2149.1) & 1141.7 (633.4-1649.9) & 1527.4 (1391.6-1663.1) & Kruskal-Wallis test & 0.111 \\
        \textbf{性别} &  &  &  & Fisher's test & 0.82 \\
        \multicolumn{1}{r}{女} & 3(75\%) & 2(\%67) & 9(\%53) &  & \\
        \multicolumn{1}{r}{男} & 1(25\%) & 1(\%33) & 8(\%47) &  & \\
        \textbf{诊断时年龄(天)} & 16(11-19) & 12(10-22) & — & Wilcoxon rank-sum test & 0.629 \\
        \textbf{住院时长(天)} & 54.3 (13.5-95.0) & 60.0 (24.8-95.2) & 32.9 (26.3-39.5) & Kruskal-Wallis test & 0.046 \\
        \textbf{标本总数} & 46 & 42 & 103 & — & — \\ \bottomrule
    \end{tabular}
  \end{table}


  使用Illumina MiSeq高通量测序、优化后,得到2,749,518条16s rRNA基因序列,优化碱基数目共1223776595,优化平均序列长度为445bp。

  \subsection{分类学分析}
  对于测序数据进行OTU聚类,采用RDP classifier贝叶斯算法对97\%相似水平的OTU代表序列进行分类学分析,并分别在各个分类水平(从域(domain)到属(genus))统计各样本的群落组成,比对silva数据库得出各群落数量。
