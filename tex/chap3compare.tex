%# -*- coding: utf-8-unix -*-
% !TEX program = xelatex
% !TEX root = ../thesis.tex
% !TEX encoding = UTF-8 Unicode
%%==================================================
%% chap3compare.tex for SJTU Master Thesis
%%==================================================

%\bibliographystyle{sjtu2}%[此处用于每章都生产参考文献]
\chapter{儿科肠道炎症性疾病与肠道菌群研究}
\label{chap:compare}

\section{引言}
\section{材料与方法}
  \subsection{伦理}
  该研究经上海市儿童医学中心伦理联合委员会批准,上海交通大学医学院(SCMCIRB-K2013022)。 在采集每名患儿的粪便样本之前,取得其监护人理解、同意并于知情同意书上签字。
  \subsection{研究对象}
  于2017年3月至2018年6月间,上海儿童医学中心新生儿重症监护病房(NICU)的NEC患儿、外科病房的HD、HAEC患儿和上海市儿童医院消化科病房的IBD患儿,
    \subsubsection{NEC入选及排除标准}
      \paragraph{入组标准}
      胎龄小于34周,出生体重不低于950g的住院患儿。
      \paragraph{排除标准}
      1)新生儿早发型败血症,2)肝脏疾病,3)肾功能损害(Cr>88μM),4)存在先天肠道发育异常,5)需要进行大型胸部或腹部手术(男性包皮环切术或动脉导管结扎除外),6)预计使用肠外营养(Parental Nutrition, PN)支持供应超过50%的每日卡路里摄入量,且时间超过4天,7)静脉注射抗生素(除头孢噻肟、哌拉西他唑和甲硝唑),8)有口服抗生素史,9)有血便史,10)日龄大于五天者
    \subsection{HD和HAEC入选及排除标准}
      \paragraph{入选标准} 1)病理确诊HD对患儿,确诊HAEC的患儿;2)需要接受手术治疗HD的患儿;3)年龄0~4岁。
      \paragraph{排除标准} 1)合并其它疾病或畸形(如先天性心脏病、唐氏综合症等),2)入组前一周内接受过抗生素,3)入组前三个月内接受过益生菌制剂。
    \subsection{IBD入选及排除标准}
      \paragraph{入选标准} 根据儿童和青少年炎症性肠病的诊断标准指南\cite{levine2014espghan},初次诊断儿童克罗恩病和溃疡性结肠炎的患儿。
      \paragraph{排除标准} 曾经接受过肠切除手术的患儿。
  \subsection{诊断标准}
    坏死性小肠结肠炎(NEC)诊断和分级:关注每名患儿入院后的健康情况,评估并记录一般情况全身,腹部平片报告结果等;根据“改良Bell分级标准”\cite{bell1978neonatal}:II期,伴有放射性肠扩张,肠梗阻,肠道积气和/或腹部压痛,和/或轻度代谢性酸中毒,肠鸣音减弱,血小板减少症。

    先天性巨结肠(HD)诊断标准为Elhalaby临床诊断标准\cite{elhalaby1995enterocolitis}和Teitelbaum病理诊断标准。

    炎症性肠病(IBD)的准确诊断:基于病史,体格检查和实验室检查,食管胃十二指肠镜检查(EGD)和组织学的回肠结肠镜检查以及小肠镜的组合。



  \subsection{主要实验室试剂及仪器}
  参照\ref{主要实验室试剂及仪器}
  \subsection{粪便标本采集方法}
  参照\ref{粪便标本采集方法}
  \subsection{标本总DNA提取}
  参照\ref{标本总DNA提取}
  \subsection{总DNA 16s rDNA V3-V4可变区片段的扩增}
  参照\ref{总DNA 16s rDNA V3-V4可变区片段的扩增}
  \subsection{荧光定量}
  参照\ref{荧光定量}
  \subsection{Illumina Miseq下一代高通量测序}
  参照\ref{Illumina Miseq下一代高通量测序}
  \subsection{原始数据处理}
  参照\ref{原始数据处理}
  \subsection{统计学方法}
  参照\ref{统计学方法}
